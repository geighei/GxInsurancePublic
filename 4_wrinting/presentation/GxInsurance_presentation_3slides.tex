%-----------------------------------------------------------------------------------------------------------------%
%\documentclass[10pt,compress,handout]{beamer}             %For printable version
\documentclass[10pt,compress,xcolor=dvipsnames]{beamer}    %For presentation version
%\setbeameroption{show notes}
%\setbeamercovered{transparent}

%% ----- Theme for projector display version --------------
\mode<presentation> {
%    \usetheme{Warsaw} %standard topbars
%    \usetheme{Berlin} %
%    \usetheme{Goettingen} %side bars
%NONE: %cleaner and more space

      \definecolor{Verde}{RGB}{05,130,25}
      \definecolor{Verde2}{RGB}{34,139,34}
	  \definecolor{mygreen}{RGB}{0,73,44}

%    \usecolortheme{dolphin}
%    \usecolortheme{beaver}f

%    \usecolortheme{seahorse}  %%bars at the top
%    \setbeamercolor{structure}{fg=Verde, bg=Verde}

%    \usecolortheme[named=Green]{structure}
%    \usecolortheme[named=ForestGreen]{structure}
%    \usecolortheme[named=OliveGreen]{structure}
    \usecolortheme[named=Verde]{structure}

    %\useinnertheme{circles} %%must use \usepackage{textcomp} otherwise Latex Warning
    \useinnertheme{rounded} %%rectangles, circles, rounded, inmargin,
    \useoutertheme{split} %%puts sections on the upper-left hand side and subsections in upper-rhs; name and title on the lower (watch out for page nums)

%    \setbeamercolor*{palette primary}{use=structure,fg=white,bg=Verde}    %%upper and lower right hs
%    \setbeamercolor*{palette secondary}{use=structure,fg=black,bg=Verde}
%    \setbeamercolor*{palette tertiary}{use=structure,fg=black,bg=blue}
%    \setbeamercolor*{palette quaternary}{use=structure,fg=black,bg=Verde}   %%upper and lower left hs

\setbeamertemplate{navigation symbols}{}  %%Get rid of the navigation symbols at the bottom
%\setbeamertemplate{footline}[page number]{} %%Show the page number at the bottom of the slides instead of name and title
%\setbeamertemplate{footline}[frame number]{} %%Show the frame number (not counting pauses) at the bottom of the slides instead of name and title
%\setbeamertemplate{footline}[text line]{\parbox{\linewidth}{\vspace*{-8pt}\insertshortauthor  \hfill  \insertpagenumber }} %/ \insertpresentationendpage

}  %%END OF MODE PRESENTATION

%% -------------- Packages ------------------------------ %%
%\pdfoptionpdfminorversion=6 %some attachments are PDF version 1.6 and I get an error
\usepackage[english]{babel}

\usepackage{amsfonts}
\usepackage{amsmath}
\usepackage{amssymb}
\usepackage{amsthm}
\usepackage{mathtools}
\usepackage{graphicx}
\graphicspath{{./include/}}% where figures are saved  %%must \usepackage{graphicx}
\usepackage{comment,footmisc,pdflscape,array,booktabs}
\usepackage{threeparttable}
\usepackage{balance}
\usepackage{lettrine}
\usepackage{type1cm}
%\usepackage{subfigure}
\usepackage{bbm}
\usepackage{multirow}
\usepackage{chngcntr}
\usepackage{multicol}
\usepackage{xr}
\usepackage{ulem}
\usepackage{color}
\usepackage{wrapfig}
\usepackage{tabularx}
\usepackage{booktabs}
\usepackage{dcolumn}

%\usepackage{color}
%\usepackage{booktabs}
\usepackage{hyperref}
%\usepackage{multimedia}
%\usepackage{animate}
\usepackage{multirow}
\usepackage{multicol}
\usepackage{bigstrut} %creates some space next to lines in table to improve appearance
\usepackage{rotating}
\usepackage{setspace}
%\usepackage{caption} %captions for the figures
\usepackage{subcaption} %allows subfigures and subcaptions
\captionsetup{compatibility=false}
\usepackage{url}
\usepackage{lmodern}  %%allowing font sizes at arbitrary sizes // removes LaTeX Font Warning: Font shape `OT1/cmss/m/n' in size <4> not available

%\usetikzlibrary{shapes,arrows,trees,snakes,positioning,fit,shadows}
\renewcommand\textbullet{\ensuremath{\bullet}}

% counter for examples
\newcounter{ex}
\setcounter{ex}{1}
\newcommand{\nbex}{\arabic{ex}\addtocounter{ex}{1}}

%figures
\renewcommand{\figurename}{Figure~\arabic{figure}}
\renewcommand{\tablename}{Table~\arabic{table}}

% math symbols
\newcommand{\Nor}[2]{N\!\left(#1, #2\right)}    	% normal distribution
\newcommand{\indp}{\perp\!\!\!\perp} 			% independence
\newcommand{\1}[1]{\mathrm{1\hspace*{-2.5pt}l}[#1]}	% indicator
\newcommand{\Var}{\mathrm{Var}}
\newcommand{\Cov}{\mathrm{Cov}}
\newcommand{\Expect}{{\rm I\kern-.3em E}}

\DeclareMathOperator{\exer}{\mathnormal{E}} %exercise


%% ----- The next 3 lines allow natbib to work properly in a Beamer Document --
\makeatletter
\def\newblock{\beamer@[EMAIL PROTECTED]}
\makeatother

%%------ New commands ---- %%
\newcommand\FontTwelve{\fontsize{12}{7.2}\selectfont} %%Reduces the font sizes of one slide only. FontSize=10, Baselineskyp=7.2
\newcommand\FontTen{\fontsize{10}{7.2}\selectfont} %%Reduces the font sizes of one slide only. FontSize=10, Baselineskyp=7.2
\newcommand\FontEight{\fontsize{8}{7.2}\selectfont} %%Reduces the font sizes of one slide only.
\newcommand\FontSix{\fontsize{6}{7.2}\selectfont} %%Reduces the font sizes of one slide only.

\hypersetup{
   colorlinks = true,
   citecolor = {Verde},
   linkcolor = {Verde},
   menucolor = {Verde},
   filecolor = {Verde},
 }
























%%%%%%%%%%%%%%%%%%%%%%%%%%%%%%%%%% TITLE %%%%%%%%%%%%%%%%%%%%%%%%%%%
\title[GxInsurance]{\textbf{Moral Hazard Heterogeneity: Genes, Health Insurance, and Smoking Decisions after a Health Shock}}
\author[Pietro, Laura]{Pietro Biroli and Laura Zwyssig}
\institute[UZH]{University of Z\"urich}
\date{\color{red} EALE-SOLE-ASSLE World Conference \\ \color{blue}{June 25, 2020}}

\begin{document}

\maketitle

%%------------------------------------------------------------------------%
%%% INTRODUCTION %
%%------------------------------------------------------------------------%
\section{Research Question}
%--------------------------------------------------------------%
\begin{frame}
\frametitle{Genes, health insurance, smoking choices}

%%%What is the question I want to answer?
\textbf{What}:
	\begin{itemize}
		%\item Understand determinants of health behaviors
		\item Does reaction to a negative health shock depend on health insurance and genetic differences?
	\end{itemize}

  \vspace{2ex}
 % \pause

%%%How do I answer the question
\textbf{How}:
\begin{itemize}
  \item Focus on smoking behavior following a cardio-vascular health shock
  \item Identification: US adults who receive free health insurance coverage after 65 (Medicare)
  \item Compare high and low genetic predisposition to smoking (G$\times$E)
\end{itemize}

  \vspace{2ex}
%  \pause

%%%Why do we care?
\textbf{Why}:
\begin{itemize}
		\item Interplay between financial and biological constraints:
		\begin{itemize}
			\item Health insurance buffers financial consequences of health shocks
			\item Genetic predispositions influence behavioral responses
		\end{itemize}
%		\item Use genes to better understand health choices
%		\begin{itemize}
			\item Understand who is at the margin
%		\end{itemize}
		\item Individualized health insurance profiles?
\end{itemize}
\end{frame}



%--------------------------------------------------------------%
\section{Regression}
%--------------------------------------------------------------%
\begin{frame}
\frametitle{Regression results}

\hspace{-5ex}
\begin{minipage}{.49\textwidth}
	\begin{table}[ht]
		\caption{Effect of the health shock} %by timing and PGS
		\small\resizebox{0.65\textheight}{!}{
		% latex table generated in R 4.0.2 by xtable 1.8-4 package
% 
\begin{tabular}{lll}
  \toprule
  \multicolumn{3}{c}{ \textbf{Effect of health shock on smoking probability}} \\
 \midrule
 & Low PGS & High PGS \\ 
   \midrule
Pre 65 & -0.165** & -0.108 \\ 
   & (0.069) & (0.083) \\ 
  Post 65 & 0.09*** & -0.13 \\ 
   & (0.026) & (0.089) \\ 
   \toprule \multicolumn{3}{c}{ \textbf{Effect of health insurance on effect of health shock}} \\
 \midrule
 & Low PGS & High PGS \\ 
   \midrule
Post 65 - Pre 65 & 0.256*** & -0.023 \\ 
   & (0.079) & (0.121) \\ 
   \toprule \multicolumn{3}{c}{ \textbf{Differential effect of health insurance by genetic group}} \\
 \midrule
 & High PGS  & - low PGS \\ 
   \midrule
Post 65 - Pre 65 & -0.279* &  \\ 
   & (0.144) &  \\ 
   \bottomrule
\end{tabular}

		}

		\vspace{1ex}

		{\raggedright \tiny \textit{Notes:} Summary of the effect of the shock on smoking for those who were uninsured before 65, stratified by timing of the shock (before vs. after 65) and genetic group (high vs. low PGS) \par}
	\end{table}
\end{minipage}
%
\hspace{4ex}
%
\begin{minipage}{.49\textwidth}
	\vspace{5ex}
	\begin{figure}[hbtp]
	%\caption{Raw cessation rate by group}
	\centering
	\includegraphics[height=0.65\textheight]{../../3_output/shock_effects/main_6070_100_cvplot.png}
	\label{fig:maincoeffplot}
	\end{figure}
	\hyperlink{frame:reg}{\beamergotobutton{regression equation}}
	\hyperlink{frame:fullreg}{\beamergotobutton{all coefficients}}
\end{minipage}

\end{frame}


%--------------------------------------------------------------%
\section{Conclusion}
%--------------------------------------------------------------%

%--------------------------------------------------------------%
\begin{frame} \frametitle{Conclusion}
\begin{itemize}
	\item Results:
	\begin{itemize}
		\item Health shock when uninsured $\Rightarrow$ less smoking...
		\item ... but \textit{only} for low PGS.
		\item Effect size is quite sizable (36.7 pp)
	\end{itemize}

	\vspace{3ex}

	\item Interaction between financial and ``biological'' constraints:
	\begin{itemize}
		\item Health insurance buffers financial consequences of health shocks
		\item Genetic predisposition to smoking mutes this effect (lower elasticity)
	\end{itemize}

	\item[$\Rightarrow$] Environment \emph{and} genes jointly influence healthy behaviors

	\vspace{3ex}

	\item Biological foundation of heterogeneity in \textbf{moral hazard} \cite{Einav2013}
	\item[$\Rightarrow$] Biological predispositions can tell a story about choices and economic fundamentals

\end{itemize}
\end{frame}

%=================================================================================
\begin{frame}
\begin{center}
{\Huge \color{Verde}{Thank you}}
\end{center}
\end{frame}




%--------------------------------------------------------------%
\appendix
\section{Appendix}
\subsection{Genes for Econ}

%=================================================================================

\begin{frame}{Genetics for social scientists} \label{frame:genetics}

    \begin{minipage}{.7\textwidth}
\begin{itemize}
	\item Human genome: series of 3 billion letter pairs (A,G,T,C)
	\item Genetic variants: one-letter changes across individuals (single nucleotide polymorphisms, SNPs)
	\item About 4-5m SNPs \cite{1000Genomes2015}
	\item Genome-wide association studies (GWAS) have identified genome-wide significant relationships between specific SNPs and health behaviors
	\item We use SNPs identified from large, replicated GWAS to create summarized genetic scores to study gene-by-SES interplay \hyperlink{fig:manhattan}{\beamergotobutton{hits}}
\end{itemize}
    \end{minipage}
    \begin{minipage}{.2\textwidth}
      \includegraphics[scale=0.4]{SNP2}
    \end{minipage}

\end{frame}

%=================================================================================

\begin{frame}{Polygenic Scores}
Construct index of genetic propensity for trait $Y$:

\begin{align}
PGS_i = \sum_{j=1}^{J} W_j G_{ij},
\end{align}

where $G_{ij}$ is the genotype for individual $i$ at SNP $j$, and the weight $W_j$ is the OLS association between SNP $j$ and outcome $Y$.

Normalized to have mean zero and standard deviation one.

\begin{figure}[hbtp]
\centering
\includegraphics[height=0.3\textheight]{Belsky2013smokingPGS}
\end{figure}

\cite{Belsky2013smoke}

\end{frame}

%=================================================================================

\begin{frame}{Polygenic Scores}
In our setting \\
Data = Health and Reteriment Study \\
Y = Pr(Smoking) \\
$W_j$ = taken from \cite{GSCAN2019gwas}

\begin{figure}[hbtp]
\centering
\includegraphics[height=0.7\textheight]{../../3_output/make_histograms/density_6070_smooth.png}
\end{figure}

\end{frame}
%%=================================================================================

\begin{frame}{Why should we care?}\label{frame:why}

\begin{itemize}
\item Genes cannot be changed ...
\item ... but \textbf{environment} can!
\item Interplay between genes (bio) and environment (econ) is essentially everywhere \cite{Rutter2006}

\vspace{3ex}

\item Understand differential effects of genetic predisposition based on environment (G$\times$E)
\begin{enumerate}
	\item[--] shed light on pathways and mechanisms
	\item[--] provide measure of essential heterogeneity
	\item[--] revisit old econ concept with new lens
\end{enumerate}

\vspace{3ex}

\item Empirics: genes are cheaper to measure and more and more available  \hyperlink{frame:dollargenome}{\beamergotobutton{}}
\item Theory: biologically founded model
%Use theory to think through issues of causality, make predictions, and guide understanding
%%aimed at understanding the various causal pathways and mechanisms

\end{itemize}
\end{frame}

\subsection{Empirical Analysis}
%--------------------------------------------------------------%
%\section{Empirical Analysis}
%--------------------------------------------------------------%

\begin{frame}\label{frame:data}
\frametitle{Data}%{Analytic Sample}

\begin{itemize}
	\item Use HRS data: US-representative survey of 50+ (1992-2014)
	\item Final sample size = $5,541$
	\begin{itemize}
		\item Ever smoked $\geq 100$ cigarettes (at baseline)
		\item Ages: 60-70
		\item Observed at least 2 waves
		\item European descendants
		\item Non-missing smoking, PGS, insurance, health shock
	\end{itemize}
\end{itemize}

\end{frame}


%--------------------------------------------------------------%
\begin{frame}\frametitle{Variables}\label{frame:vars}

\begin{itemize}
	\item Outcome \textit{\color{Verde} Y} = current smoking status $\in \{0,1\}$
	\begin{itemize}
		\item Self reported
		\item Also calculate cessation rates (smoking in previous but not current wave)
	\end{itemize}

	\smallskip

	\item High PGS \textit{\color{Verde} g} = above 33$^{rd}$ in PGS for smoking initiation
	\begin{itemize}
		\item Use \cite{GSCAN2019gwas} for weights
	\end{itemize}

	\smallskip

	\item Health {\color{Verde} shock} = first diagnosis of acute cardiovascular condition
	\begin{itemize}
		\item Heart problem: heart attack, coronary heart disease, angina, congestive heart
failure, or other heart problems
		\item or Stroke: transient ischemic attack
	\end{itemize}

	\smallskip

	\item {\color{Verde} Uninsured}: self-reported coverage
	\begin{itemize}
		\item Pre 65 uninsured: never report being covered by medical insurance
		\item Post 65 everyone insured: eligible for Medicare
	\end{itemize}
\end{itemize}

\end{frame}

%--------------------------------------------------------------%
\begin{frame}{Identification}\label{frame:id}

Diff in response to health shock before and after 65


Main assumption: \textbf{timing} of health shock is exogenous \cite{Marti2017,Card2009medicare}


\begin{figure}[hbtp]
%\caption{Percentage of Reported Health Shocks by Age}
\centering
\includegraphics[height=0.75\textheight]{../../3_output/cv_prob/main_6070plot_pgs.png}
%\includegraphics[height=0.8\textheight]{../../3_output/cv_prob/RD_age65_CVshock.png}
\label{fig:cv_prob}
\end{figure}

%\begin{figure}[!tbp]
%	\begin{center}
%		\caption{Percentage of Reported Health Shocks by Age \label{fig:cv_prob}}
%		\subfigure[\textsf{Average by year}]{\includegraphics[width=0.4\textheight]{../../3_output/cv_prob/main_6070plot_pgs.png} }
%		\hfill
%		\subfigure[\textsf{RDD}]{\includegraphics[width=0.4\textheight]{../../3_output/cv_prob/RD_age65_CVshock.png} }
%		%\floatfoot{ \vspace{-0.8cm} \\
%		%	\textsf{Low PGS: PGS $\leq$ median PGS. High PGS: PGS $>$ median PGS. \\
%		%		Bars show 95\% confidence intervals. Dotted bars indicate that it was unknown if reported shocks occurred before or after age 65. \\
%		%		Data used: HRS waves 1-12, restricted to observations with age between 60 and 70 years and non-missing smoking status.}}
%	\end{center}
%\end{figure}
%
\end{frame}

%--------------------------------------------------------------%
\begin{frame}{Identification}\label{frame:id2}


\begin{figure}[hbtp]
%\caption{Percentage of Reported Health Shocks by Age}
\centering
\includegraphics[width=0.9\textheight]{../../3_output/over_time/graph_6070cvrdd_agebypgs.png}
%\includegraphics[height=0.9\textheight]{../../3_output/rd_plots/graph_6070_rdplot_cv.png}
%\includegraphics[height=0.8\textheight]{../../3_output/cv_prob/RD_age65_CVshock.png}
\label{fig:cv_prob_rdd}
\end{figure}

\end{frame}

%--------------------------------------------------------------%
\begin{frame}
\frametitle{Summary statistics}\label{frame:sumstats}
\begin{table}[ht]
	\caption{Descriptive Statistics for Full Analytic Sample and Stratified by Genetic Group}
	\small\resizebox{\textwidth}{!}{
		% latex table generated in R 4.0.2 by xtable 1.8-4 package
% 
\begin{tabular}{llll}
  \toprule
\textbf{ } & \textbf{ All } & \textbf{ Low PGS } & \textbf{ High PGS } \\ 
  \midrule
 & Mean (SD) & Mean (SD) & Mean (SD) \\ 
   \midrule
Age (baseline) & 61.15 (1.87) & 61.17 (1.9) & 61.15 (1.85) \\ 
  Smoking PGS & 0.1 (0.99) & -0.96 (0.51) & 0.63 (0.71) \\ 
  No. waves present & 4.45 (1.36) & 4.47 (1.35) & 4.44 (1.37) \\ 
   & \% & \% & \% \\ 
  Female & 49.86 & 46.47 & 51.55 \\ 
  Smoking (baseline) & 29.5 & 26.97 & 30.76 \\ 
  Persistently uninsured & 5.88 & 5.46 & 6.09 \\ 
  CV health shock & 12.28 & 11.43 & 12.71 \\ 
   \midrule
No. of individuals & 5854 & 1943 & 3911 \\ 
   \midrule
Person-year observations & 26022 & 8676 & 17346 \\ 
  \end{tabular}

	}
\end{table}

\hyperlink{frame:sumstats2}{\beamergotobutton{Sumstats2}}

\end{frame}

%--------------------------------------------------------------%
\begin{frame}
\frametitle{Summary statistics, by health shock timing}\label{frame:sumstats2}

\begin{table}[ht]
	\caption{Descriptive Statistics for Subset of the Analytical Sample with a Health Shock, Stratified by Timing fo the Shock and Genetic Group}
	\small\resizebox{0.7\textwidth}{!}{
		% latex table generated in R 4.0.2 by xtable 1.8-4 package
% 
\begin{tabular}{llll}
  \toprule
\textbf{  } & \textbf{ Low PGS } & \textbf{ High PGS } & \textbf{ P value } \\ 
  \midrule
Shock at ages 60-64 &  &  &  \\ 
   \midrule
 & Mean (SD) & Mean (SD) &  \\ 
  Age (baseline) & 60.49 (0.57) & 60.46 (0.62) & 0.67 \\ 
  Smoking PGS & -0.97 (0.54) & 0.63 (0.68) & 0.00 \\ 
  Years of education & 12.17 (3.44) & 12.15 (3.12) & 0.95 \\ 
  Income (nominal \$ 1000) & 19.61 (27.99) & 18.9 (30.1) & 0.83 \\ 
  No. waves present & 4.65 (1.31) & 4.59 (1.31) & 0.66 \\ 
   & \% & \% &  \\ 
  Female & 48.21 & 45.26 & 0.60 \\ 
  Smoking (baseline) & 30.36 & 37.23 & 0.19 \\ 
  Persistently uninsured & 4.46 & 6.57 & 0.39 \\ 
  Avg. cessation rate (baseline smokers) & 11.95 & 12.09 & 0.96 \\ 
  No. of individuals & 112 & 274 &  \\ 
   \midrule
No. of Person-year individuals & 521 & 1257 &  \\ 
   \midrule
Shock at ages 67-70 &  &  &  \\ 
   & Mean (SD) & Mean (SD) &  \\ 
  Age (baseline) & 61.16 (1.9) & 61.01 (1.22) & 0.44 \\ 
  Smoking PGS & -0.94 (0.48) & 0.76 (0.8) & 0.00 \\ 
  Years of education & 12.73 (3.15) & 12.42 (3.07) & 0.40 \\ 
  Income (nominal \$ 1000) & 17.75 (27.24) & 16.1 (20.41) & 0.58 \\ 
  No. waves present & 5.13 (1.05) & 5.13 (0.75) & 0.99 \\ 
   & \% & \% &  \\ 
  Female & 40.91 & 47.98 & 0.22 \\ 
  Smoking (baseline) & 28.18 & 34.98 & 0.21 \\ 
  Persistently uninsured & 8.18 & 6.28 & 0.54 \\ 
  Avg. cessation rate (baseline smokers) & 9.94 & 10.83 & 0.75 \\ 
   \midrule
No. of individuals & 110 & 223 &  \\ 
  No. of Person-year individuals & 564 & 1143 &  \\ 
  \end{tabular}

	}
\end{table}

%\begin{figure}[hbtp]
%\caption{Raw cessation rate by group}
%\centering
%\includegraphics[height=0.8\textheight]{table2.png}
%\label{tab:sumstat2}
%\end{figure}

\hyperlink{frame:sumstats}{\beamergotobutton{back}}
\end{frame}


%--------------------------------------------------------------%
\begin{frame}
\frametitle{Smoking rates decrease with age}
Share of smokers over different ages, split by PGS

\begin{figure}[hbtp]
\centering
\includegraphics[height=0.8\textheight]{../../3_output/over_time/graph_6070smokenplot_agebypgs.png}
\label{fig:}
\end{figure}

\end{frame}

\subsection{Regression Results}
%--------------------------------------------------------------%
\begin{frame}\frametitle{Regression analysis}\label{frame:reg}
%\vspace{-5ex}
\begin{footnotesize}
Use HRS data, ages 60-70:
\begin{align*} \label{eq:OLS}
Y_{it}& \thinspace  = \thinspace
				\beta \thinspace shock_{it} + \gamma \thinspace post65_{it} \\
				&+\lambda_1 \thinspace  (shock_{it} \times post65_{it}) \nonumber \\
				&+ \lambda_2 \thinspace (shock_{it} \times uninsured_i) \nonumber \\
				&+\lambda_3  \thinspace (post65_{it} \times uninsured_i) \nonumber \\
				&+ \lambda_4 \thinspace (shock_{it} \times g_i) \nonumber \\
				&+\lambda_5 \thinspace (post65_{it} \times g_i) \nonumber \\
				&+ \delta_1 \thinspace (shock_{it} \times post65_{it} \times uninsured_i) \nonumber \\
				&+ \delta_2 \thinspace (shock_{it} \times uninsured_i \times g_i) \nonumber \\
				&+ \delta_3 \thinspace (shock_{it} \times post65_{it} \times g_i) \nonumber \\
				&+ \delta_4 \thinspace (post65_{it} \times uninsured_i \times g_i) \nonumber\\
				&+ \zeta \thinspace (shock_{it} \times post65_{it} \times uninsured_i \times g_i) \nonumber \\
				&+ \sum_{a=1}^3 \phi_a \thinspace age_{it}^{a} + \eta_i + \tau_t + \varepsilon_{it} \nonumber
\end{align*}

Current smoking status ($Y$) regressed on the full set of interactions between the indicators for the health shock ($shock$), being uninsured pre-65 ($uninsured$), Medicare eligibility ($post65$), and high polygenic risk for smoking ($g$).

Controlling for age + individual and time F.E.
\end{footnotesize}

\hyperlink{fig:maincoeffplot}{\beamergotobutton{back}}
\end{frame}


%--------------------------------------------------------------%
\begin{frame}
\frametitle{Regression results} \label{frame:fullreg}

\begin{table}[ht]
	%\caption{Coefficients from Estimating the Linear Probability Model in Equation (2) Using OLS}
	\resizebox{0.9\textheight}{!}{
		
% Table created by stargazer v.5.2.2 by Marek Hlavac, Harvard University. E-mail: hlavac at fas.harvard.edu
% Date and time: Wed, Dec 09, 2020 - 11:00:44 PM
% Requires LaTeX packages: dcolumn 
\begin{tabular}{@{\extracolsep{0pt}}lD{.}{.}{-3} D{.}{.}{-3} D{.}{.}{-3} D{.}{.}{-3} D{.}{.}{-3} } 
\\[-1.8ex]\hline 
\hline \\[-1.8ex] 
 & \multicolumn{5}{c}{\textit{Dependent variable:}} \\ 
\cline{2-6} 
\\[-1.8ex] & \multicolumn{5}{c}{Smoking status} \\ 
\\[-1.8ex] & \multicolumn{1}{c}{(1)} & \multicolumn{1}{c}{(2)} & \multicolumn{1}{c}{(3)} & \multicolumn{1}{c}{(4)} & \multicolumn{1}{c}{(5)}\\ 
\hline \\[-1.8ex] 
 Health Shock & -0.027 & -0.028 & -0.027 & -0.046^{*} & -0.045^{*} \\ 
  & (0.040) & (0.040) & (0.040) & (0.024) & (0.024) \\ 
  & & & & & \\ 
 Post-65 & -0.064^{***} & -0.020 & -0.020 & -0.010 & -0.009 \\ 
  & (0.008) & (0.012) & (0.012) & (0.008) & (0.008) \\ 
  & & & & & \\ 
 Uninsured & 0.169^{***} & 0.169^{***} & 0.169^{***} &  &  \\ 
  & (0.027) & (0.027) & (0.027) &  &  \\ 
  & & & & & \\ 
 High PGS & 0.033^{***} & 0.033^{***} & 0.033^{***} &  &  \\ 
  & (0.012) & (0.012) & (0.012) &  &  \\ 
  & & & & & \\ 
 Shock $\times$ Post-65 & 0.008 & 0.018 & 0.016 & 0.026 & 0.025 \\ 
  & (0.056) & (0.056) & (0.056) & (0.032) & (0.032) \\ 
  & & & & & \\ 
 Shock $\times$ Uninsured & -0.193 & -0.187 & -0.188 & -0.118 & -0.120^{*} \\ 
  & (0.184) & (0.185) & (0.184) & (0.073) & (0.073) \\ 
  & & & & & \\ 
 Post-65 $\times$ Uninsured & 0.068 & 0.067 & 0.067 & -0.052^{*} & -0.052^{*} \\ 
  & (0.048) & (0.048) & (0.048) & (0.031) & (0.031) \\ 
  & & & & & \\ 
 Shock $\times$ High PGS & 0.028 & 0.029 & 0.026 & 0.015 & 0.015 \\ 
  & (0.049) & (0.049) & (0.049) & (0.029) & (0.029) \\ 
  & & & & & \\ 
 Post-65 $\times$ High PGS & -0.0003 & -0.0005 & -0.0005 & -0.005 & -0.005 \\ 
  & (0.010) & (0.010) & (0.010) & (0.008) & (0.008) \\ 
  & & & & & \\ 
 Shock $\times$ Post-65 x Uninsured & 0.343 & 0.336 & 0.336 & 0.230^{***} & 0.230^{***} \\ 
  & (0.256) & (0.256) & (0.255) & (0.086) & (0.085) \\ 
  & & & & & \\ 
 Shock $\times$ Uninsured $\times$ High PGS & 0.184 & 0.182 & 0.183 & 0.041 & 0.042 \\ 
  & (0.219) & (0.220) & (0.219) & (0.112) & (0.112) \\ 
  & & & & & \\ 
 Shock $\times$ Post-65 $\times$ High PGS & -0.047 & -0.047 & -0.045 & -0.078^{*} & -0.078^{*} \\ 
  & (0.068) & (0.068) & (0.068) & (0.042) & (0.042) \\ 
  & & & & & \\ 
 Post-65 $\times$ Uninsured $\times$ High PGS & -0.116^{*} & -0.117^{*} & -0.117^{*} & 0.042 & 0.042 \\ 
  & (0.061) & (0.062) & (0.062) & (0.036) & (0.036) \\ 
  & & & & & \\ 
 Shock $\times$ Post-65 $\times$ Uninsured $\times$ High PGS & -0.279 & -0.278 & -0.278 & -0.199 & -0.200 \\ 
  & (0.311) & (0.312) & (0.311) & (0.151) & (0.150) \\ 
  & & & & & \\ 
\hline \\[-1.8ex] 
Age & & Yes & Yes & Yes & Yes  \\
Year FE & &              & Yes &              & Yes  \\
Individual FE   & &              &              & Yes & Yes  \\
 \hline \\[-1.8ex]
Observations & \multicolumn{1}{c}{26,022} & \multicolumn{1}{c}{26,022} & \multicolumn{1}{c}{26,022} & \multicolumn{1}{c}{26,022} & \multicolumn{1}{c}{26,022} \\ 
R$^{2}$ & \multicolumn{1}{c}{0.015} & \multicolumn{1}{c}{0.017} & \multicolumn{1}{c}{0.017} & \multicolumn{1}{c}{0.823} & \multicolumn{1}{c}{0.823} \\ 
\hline 
\hline \\[-1.8ex] 
\end{tabular} 

	}
\end{table}

\hyperlink{fig:maincoeffplot}{\beamergotobutton{back}}

\end{frame}

%--------------------------------------------------------------%
\begin{frame}
\frametitle{Meaning of OLS coefficients} \label{frame:OLSmath}
From estimating equation \ref{eq:OLS} we get the following:
\hyperlink{fig:maincoeffplot}{\beamergotobutton{back}}

\begin{footnotesize}
\begin{align}
\phantom{E\left[ Y_{it}| post65_{it}=1, g_i=1, shock_{it}=1,uninsured_i=1\right]}
&\begin{aligned}
	\mathllap{E\left[ Y_{it}| post65_{it}=0, g_i=0, shock_{it}=1,uninsured_i=1\right]} &=\beta+\lambda_2
\end{aligned}\\
&\begin{aligned}
	\mathllap{E\left[ Y_{it}| post65_{it}=1, g_i=0, shock_{it}=1,uninsured_i=1\right]} &=\beta+\gamma+\lambda_1+\lambda_2\\ &+\lambda_3+\delta_1
\end{aligned}\\
&\begin{aligned}
	\mathllap{E\left[ Y_{it}| post65_{it}=0, g_i=1, shock_{it}=1,uninsured_i=1\right]} &=\beta+\lambda_2+\lambda_4+\delta_2
\end{aligned}\\
&\begin{aligned}
	\mathllap{E\left[ Y_{it}| post65_{it}=1, g_i=1, shock_{it}=1,uninsured_i=1\right]} &=\beta+\gamma+\lambda_1+\lambda_2\\
	&+\lambda_3+\lambda_4+\lambda_5+\delta_1+\delta_2\\
	&+\delta_3+\delta_4+\xi
\end{aligned}
\end{align}

Then the first two differences yield:

\begin{align}
(2)-(1)&=\gamma+\lambda_1+\lambda_3+\delta_1\\
(4)-(3)&=\gamma+\lambda_1+\lambda_3+\lambda_5+\delta_1+\delta_3+\delta_4+\xi
\end{align}

And finally the diff-in-diff (G$\times$E) is identified by:
\begin{align}
	(6)-(5)&=\lambda_5+\delta_3+\delta_4+\xi
\end{align}
\end{footnotesize}

\end{frame}

%--------------------------------------------------------------%
\begin{frame}
\frametitle{Effect of the shock on the outcomes} \label{frame:shockmath}
The derivative of the outcome with respect to shock is:
\hyperlink{fig:maincoeffplot}{\beamergotobutton{back}}

\begin{footnotesize}
\begin{align}
\begin{aligned}
	\frac{\partial Y_{it}}{\partial shock_{it}}&=\beta+\lambda_1post65_{it}+\lambda_2uninsured_{i}+\lambda_4g_i\\
	&+\delta_1(post65 \times uninsured_i)+\delta_2(uninsured_i \times g_i)+\delta_3(post65_{it} \times g_i)\\
	&+\xi(post65_{it} \times uninsured_i \times g_i)
\end{aligned}
\end{align}

Again, we can look at the decomposition:
\begin{align}
\phantom{E\left[ \frac{\partial Y_{it}}{\partial shock_{it}}| post65_{it}=1, g_i=1,uninsured_i=1\right]}
&\begin{aligned}
\mathllap{E\left[ \frac{\partial Y_{it}}{\partial shock_{it}}| post65_{it}=0, g_i=0,uninsured_i=1\right]} &=\beta+\lambda_2
\end{aligned}\\
&\begin{aligned}
\mathllap{E\left[ \frac{\partial Y_{it}}{\partial shock_{it}}| post65_{it}=1, g_i=0,uninsured_i=1\right]} &=\beta+\lambda_1+\lambda_2+\delta_1
\end{aligned}\\
&\begin{aligned}
\mathllap{E\left[ \frac{\partial Y_{it}}{\partial shock_{it}}| post65_{it}=0, g_i=1,uninsured_i=1\right]} &=\beta+\lambda_2+\lambda_4+\delta_2
\end{aligned}\\
&\begin{aligned}
\mathllap{E\left[ \frac{\partial Y_{it}}{\partial shock_{it}}| post65_{it}=1, g_i=1,,uninsured_i=1\right]} &=\beta+\lambda_1+\lambda_2+\lambda_4\\
&+\delta_1+\delta_2+\delta_3+\xi
\end{aligned}
\end{align}

Calculating the first two differences as above::

\begin{align}
(10)-(9)&=\lambda_1+\delta_1\\
(12)-(11)&=\lambda_1+\delta_1+\delta_3+\xi
\end{align}

And again the diff-in-diff (G$\times$E):
\begin{align}
(14)-(13)&=\delta_3+\xi
\end{align}
\end{footnotesize}

\end{frame}


%--------------------------------------------------------------%
\subsection{Confounders}
%--------------------------------------------------------------%
\begin{frame}
\frametitle{Potential interpretation problems}
\label{frame:interpret}
What else can drive this relation?
\begin{itemize}
	\item Something that jumps at 65 (besides medicare)
	\begin{itemize}
		\item Retirement and income: no sharp change at age 65 \cite{Card2008,Card2009medicare}
	\end{itemize}
\end{itemize}

\vspace{4ex}
\hspace{-4ex}
\includegraphics[width=.36\textwidth]{../../3_output/over_time/graph_6070govmrrdd_agebypgs.png}%
\includegraphics[width=.36\textwidth]{../../3_output/over_time/graph_6070retiredrdd_agebypgs.png}%
\includegraphics[width=.36\textwidth]{../../3_output/over_time/graph_6070iearnrdd_agebypgs.png}

\end{frame}

%--------------------------------------------------------------%
\begin{frame}
\frametitle{Potential Confounders}
\label{frame:Confounders}
Check for differential response to shock on possible confounders \cite{Pei2018}
\begin{itemize}
	\item Retirement 		\hyperlink{fig:retire}{\beamergotobutton{}}
	\item Individual income \hyperlink{fig:wage}{\beamergotobutton{}}
	\item Household income	\hyperlink{fig:hhinc}{\beamergotobutton{}}
	%\item Wealth 			\hyperlink{fig:wealth}{\beamergotobutton{}}
	\item Out of pocket med. expenditure   \hyperlink{fig:oome}{\beamergotobutton{}}
	%\item Marital status 	\hyperlink{fig:married}{\beamergotobutton{}}
	\item Having a partner	\hyperlink{fig:mpart}{\beamergotobutton{}}
	\item Mortality \hyperlink{fig:dead2}{\beamergotobutton{2y}} \hyperlink{fig:dead5}{\beamergotobutton{5y}}
\end{itemize}

\end{frame}

%--------------------------------------------------------------% retire
\begin{frame}
\frametitle{Retirement: not likely a Confounder}
Coefficient plot of the effect of the shock on being retired.
\begin{figure}[hbtp]
%\caption{Raw cessation rate by group}
\centering
\includegraphics[height=0.8\textheight]{../../3_output/shock_effects/retire_6070_100_cvplot.png}
\label{fig:retire}
\end{figure}
%\hyperlink{frame:robustness}{\beamergotobutton{back}}
\end{frame}

%--------------------------------------------------------------% wage
\begin{frame}
\frametitle{Wage: not likely a Confounder}
Coefficient plot of the effect of the shock on log reported earnings.
\begin{figure}[hbtp]
%\caption{Raw cessation rate by group}
\centering
\includegraphics[height=0.8\textheight]{../../3_output/shock_effects/IHSwage_6070_100_cvplot.png}
\label{fig:wage}
\end{figure}
%\hyperlink{frame:robustness}{\beamergotobutton{back}}
\end{frame}

%--------------------------------------------------------------% hhinc
\begin{frame}
\frametitle{Household Income: not likely a Confounder}
Coefficient plot of the effect of the shock on household income.
\begin{figure}[hbtp]
%\caption{Raw cessation rate by group}
\centering
\includegraphics[height=0.8\textheight]{../../3_output/shock_effects/hhinc_6070_100_cvplot.png}
\label{fig:hhinc}
\end{figure}
%\hyperlink{frame:robustness}{\beamergotobutton{back}}
\end{frame}

%%--------------------------------------------------------------% wealth
%\begin{frame}
%\frametitle{Wealth: possibly a Confounder?}
%Coefficient plot of the effect of the shock on household wealth.
%\begin{figure}[hbtp]
%%\caption{Raw cessation rate by group}
%\centering
%\includegraphics[height=0.8\textheight]{../../3_output/shock_effects/wealth_6070_100_cvplot.png}
%\label{fig:wealth}
%\end{figure}
%%\hyperlink{frame:robustness}{\beamergotobutton{back}}
%\end{frame}

%--------------------------------------------------------------% medexp
\begin{frame}
\frametitle{Medical expenditure: not likely a Confounder}
Coefficient plot of the effect of the shock on out of poket medical enxpenditure.
\begin{figure}[hbtp]
%\caption{Raw cessation rate by group}
\centering
\includegraphics[height=0.8\textheight]{../../3_output/shock_effects/medexp_6070_100_cvplot.png}
\label{fig:medexp}
\end{figure}
%\hyperlink{frame:robustness}{\beamergotobutton{back}}
\end{frame}

%%--------------------------------------------------------------% married
%\begin{frame}
%\frametitle{Married: not likely a Confounder}
%Coefficient plot of the effect of the shock on being married.
%\begin{figure}[hbtp]
%%\caption{Raw cessation rate by group}
%\centering
%\includegraphics[height=0.8\textheight]{../../3_output/shock_effects/married_6070_100_cvplot.png}
%\label{fig:married}
%\end{figure}
%%\hyperlink{frame:robustness}{\beamergotobutton{back}}
%\end{frame}

%--------------------------------------------------------------% mpart
\begin{frame}
\frametitle{Having a partner: not likely a Confounder}
Coefficient plot of the effect of the shock on having a partner.
\begin{figure}[hbtp]
%\caption{Raw cessation rate by group}
\centering
\includegraphics[height=0.8\textheight]{../../3_output/shock_effects/mpart_6070_100_cvplot.png}
\label{fig:mpart}
\end{figure}
%\hyperlink{frame:robustness}{\beamergotobutton{back}}
\end{frame}

%--------------------------------------------------------------% dead2
\begin{frame}
\frametitle{2 year mortality: not likely a Confounder}
Coefficient plot of the probability of dying within 2 years of the shock.
\begin{figure}[hbtp]
\centering
\includegraphics[height=0.8\textheight]{../../3_output/shock_effects/dead2_6070_100_cvplot.png}
\label{fig:dead2}
\end{figure}
%\hyperlink{frame:robustness}{\beamergotobutton{back}}
\end{frame}

%--------------------------------------------------------------% dead5
\begin{frame}
\frametitle{5 year mortality: not likely a Confounder}
Coefficient plot of the probability of dying within 5 years of the shock.
\begin{figure}[hbtp]
\centering
\includegraphics[height=0.8\textheight]{../../3_output/shock_effects/dead5_6070_100_cvplot.png}
\label{fig:dead5}
\end{figure}
%\hyperlink{frame:robustness}{\beamergotobutton{back}}
\end{frame}

%--------------------------------------------------------------%
\subsection[Split]{Other Characteristics}
%--------------------------------------------------------------%
\begin{frame} \label{frame:otherX}
\frametitle{Is it really genes?}
Are there other characteristics that might be driving this relationship?

\begin{itemize}
	\item Try to cut the data according to other dimensions:
	\begin{itemize}
		\item Cognitive skills 	\hyperlink{fig:cog}{\beamergotobutton{}}
		\item Conscientiousness \hyperlink{fig:consc}{\beamergotobutton{}}
		\item Risk aversion 	\hyperlink{fig:risk}{\beamergotobutton{}}
		\item Gender 			\hyperlink{fig:gender}{\beamergotobutton{}}
		\item Education 		\hyperlink{fig:edu}{\beamergotobutton{}}
		\item Income 			\hyperlink{fig:income}{\beamergotobutton{}}
	\end{itemize}

	\item Or according to other PGS:
	\begin{itemize}
		\item Cognition PGS 	\hyperlink{fig:cogPGS}{\beamergotobutton{}}
		%\item Non-cognitive PGS \hyperlink{fig:noncogPGS}{\beamergotobutton{}}
		\item Risk aversion PGS	\hyperlink{fig:riskPGS}{\beamergotobutton{}}
		%\item Education PGS		\hyperlink{fig:ea3PGS}{\beamergotobutton{}}
	\end{itemize}
\end{itemize}

\end{frame}

%--------------------------------------------------------------%
\begin{frame}
\frametitle{Split by cognitive ability}

\begin{figure}[hbtp]
\centering
\includegraphics[height=0.8\textheight]{../../3_output/shock_effects/cog_6070_100_cvplot.png}
\label{fig:cog}
\end{figure}
\hyperlink{frame:otherX}{\beamergotobutton{back}}
\end{frame}

%--------------------------------------------------------------%
\begin{frame}
\frametitle{Split by conscientiousness}

\begin{figure}[hbtp]
\centering
\includegraphics[height=0.8\textheight]{../../3_output/shock_effects/consc_6070_100_cvplot.png}
\label{fig:consc}
\end{figure}
\hyperlink{frame:otherX}{\beamergotobutton{back}}
\end{frame}

%--------------------------------------------------------------%
\begin{frame}
\frametitle{Split by risk aversion}

\begin{figure}[hbtp]
\centering
\includegraphics[height=0.8\textheight]{../../3_output/shock_effects/risk_6070_100_cvplot.png}
\label{fig:risk}
\end{figure}
\hyperlink{frame:otherX}{\beamergotobutton{back}}
\end{frame}
%--------------------------------------------------------------%
\begin{frame}
\frametitle{Split by gender}

\begin{figure}[hbtp]
\centering
\includegraphics[height=0.8\textheight]{../../3_output/shock_effects/female_6070_100_cvplot.png}
\label{fig:female}
\end{figure}
\hyperlink{frame:otherX}{\beamergotobutton{back}}
\end{frame}
%--------------------------------------------------------------%
\begin{frame}
\frametitle{Split by education}

\begin{figure}[hbtp]
\centering
\includegraphics[height=0.8\textheight]{../../3_output/shock_effects/edu_6070_100_cvplot.png}
\label{fig:edu}
\end{figure}
\hyperlink{frame:otherX}{\beamergotobutton{back}}
\end{frame}








%--------------------------------------------------------------%
\subsection{Robustness}
%--------------------------------------------------------------%

%--------------------------------------------------------------%
\begin{frame}
\frametitle{Robustness checks}
\label{frame:robustness}

Robustness checks:

\begin{itemize}
	\item Different cutoffs for high-PGS \hyperlink{fig:coeffplot25highPGS}{\beamergotobutton{PGS}}
	\begin{itemize}
		\item Still holds for lower quartile, not for highest
		\item Using continuous PGS, G$\times$E effect = -0.13 (0.11)
	\end{itemize}

	\vspace{1ex}

	\item Using the publicly available PGS (older GWAS from \cite{TAG2010}) \hyperlink{fig:oldPGS}{\beamergotobutton{PGS}}
	\begin{itemize}
		\item Similar pattern
	\end{itemize}

	\vspace{1ex}

	\item Different cutoffs for uninsured \hyperlink{fig:coeffplot66unins}{\beamergotobutton{uninsured}}
	\begin{itemize}
		\item Noisy results for uninsured only 1/3 of the times
	\end{itemize}

	\vspace{1ex}

	\item Different cutoffs for age \hyperlink{fig:coeffplot59-71}{\beamergotobutton{age}}
	\begin{itemize}
		\item Noisy results after age 72
	\end{itemize}
\end{itemize}

\end{frame}


\subsubsection{App: Robustness by Age}
%------------ ROBUSTNESS AGE RANGE-----------------------------%
%--------------------------------------------------------------%
\begin{frame}
\frametitle{Robustness: age range 59-71}
Coefficient plot of the main regression.
\begin{figure}[hbtp]
%\caption{Raw cessation rate by group}
\centering
\includegraphics[height=0.8\textheight]{../../3_output/shock_effects/robustness_5971_100_cv_5971plot.png}
\label{fig:coeffplot59-71}
\end{figure}
\hyperlink{frame:robustness}{\beamergotobutton{back}}
\end{frame}

%--------------------------------------------------------------%
\begin{frame}
\frametitle{Robustness: age range 58-72}
Coefficient plot of the main regression.
\begin{figure}[hbtp]
%\caption{Raw cessation rate by group}
\centering
\includegraphics[height=0.8\textheight]{../../3_output/shock_effects/robustness_5872_100_cv_5872plot.png}
\label{fig:coeffplot58-72}
\end{figure}
\hyperlink{frame:robustness}{\beamergotobutton{back}}
\end{frame}

%--------------------------------------------------------------%
\begin{frame}
\frametitle{Robustness: age range 57-73}
Coefficient plot of the main regression.
\begin{figure}[hbtp]
%\caption{Raw cessation rate by group}
\centering
\includegraphics[height=0.8\textheight]{../../3_output/shock_effects/robustness_5773_100_cv_5773plot.png}
\label{fig:coeffplot57-73}
\end{figure}
\hyperlink{frame:robustness}{\beamergotobutton{back}}
\end{frame}

%--------------------------------------------------------------%
\begin{frame}
\frametitle{Robustness: age range 56-74}
Coefficient plot of the main regression.
\begin{figure}[hbtp]
%\caption{Raw cessation rate by group}
\centering
\includegraphics[height=0.8\textheight]{../../3_output/shock_effects/robustness_5674_100_cv_5674plot.png}
\label{fig:coeffplot56-74}
\end{figure}
\hyperlink{frame:robustness}{\beamergotobutton{back}}
\end{frame}

\subsubsection{App: Robustness by uninsured spells}
%------------ ROBUSTNESS UNINSURED -----------------------------%
%--------------------------------------------------------------%
\begin{frame}
\frametitle{Robustness: uninsured only 2/3 of the time}
Coefficient plot of the main regression.
\begin{figure}[hbtp]
%\caption{Raw cessation rate by group}
\centering
\includegraphics[height=0.8\textheight]{../../3_output/shock_effects/robustness_607066_cvplot.png}
\label{fig:coeffplot66unins}
\end{figure}
\hyperlink{frame:robustness}{\beamergotobutton{back}}
\end{frame}

%--------------------------------------------------------------%
\begin{frame}
\frametitle{Robustness: uninsured only 1/3 of the time}
Coefficient plot of the main regression.
\begin{figure}[hbtp]
%\caption{Raw cessation rate by group}
\centering
\includegraphics[height=0.8\textheight]{../../3_output/shock_effects/robustness_607033_cvplot.png}
\label{fig:coeffplot33unins}
\end{figure}
\hyperlink{frame:robustness}{\beamergotobutton{back}}
\end{frame}

\subsubsection{App: Robustness by PGS cutoffs}
%------------ ROBUSTNESS PGS cutoff -----------------------------%

%--------------------------------------------------------------%
\begin{frame}
\frametitle{Different PGS cutoffs}

\begin{figure}[hbtp]
\centering
\includegraphics[height=0.8\textheight]{../../3_output/shock_effects/robustness_6070__3pgs_cvplot.png}
\label{fig:3pgs}
\end{figure}

\end{frame}

%--------------------------------------------------------------%
\begin{frame}
\frametitle{Robustness: high PGS if above 25$^{th}$ percentile}
Coefficient plot of the main regression.
\begin{figure}[hbtp]
%\caption{Raw cessation rate by group}
\centering
\includegraphics[height=0.8\textheight]{../../3_output/shock_effects/robustness_6070__25pt_cvplot.png}
\label{fig:coeffplot25highPGS}
\end{figure}
\hyperlink{frame:robustness}{\beamergotobutton{back}}
\end{frame}

%--------------------------------------------------------------%
\begin{frame}
\frametitle{Robustness: high PGS if above 50$^{th}$ percentile}
Coefficient plot of the main regression.
\begin{figure}[hbtp]
%\caption{Raw cessation rate by group}
\centering
\includegraphics[height=0.8\textheight]{../../3_output/shock_effects/robustness_6070__50pt_cvplot.png}
\label{fig:coeffplot50highPGS}
\end{figure}
\hyperlink{frame:robustness}{\beamergotobutton{back}}
\end{frame}

%--------------------------------------------------------------%
\begin{frame}
\frametitle{Robustness: high PGS if above 75$^{th}$ percentile}
Coefficient plot of the main regression.
\begin{figure}[hbtp]
%\caption{Raw cessation rate by group}
\centering
\includegraphics[height=0.8\textheight]{../../3_output/shock_effects/robustness_6070_100_cv_75pplot.png}
\label{fig:coeffplot75highPGS}
\end{figure}
\hyperlink{frame:robustness}{\beamergotobutton{back}}
\end{frame}

%--------------------------------------------------------------%
\begin{frame}
\frametitle{Robustness: using publicly available PGS (older GWAS)}
Coefficient plot of the main regression.
\begin{figure}[hbtp]
%\caption{Raw cessation rate by group}
\centering
\includegraphics[height=0.8\textheight]{../../3_output/shock_effects/robustness_6070__oldpgs_cvplot.png}

\label{fig:oldPGS}
\end{figure}
\hyperlink{frame:robustness}{\beamergotobutton{back}}
\end{frame}



\subsubsection{App Robustness by other PGS}
%--------------------------------------------------------------%
\begin{frame}
\frametitle{Split by cognitive ability PGS}

\begin{figure}[hbtp]
\centering
\includegraphics[height=0.8\textheight]{../../3_output/shock_effects/robustness_cogPGS_6070_100_cvplot.png}
\label{fig:cogPGS}
\end{figure}
\hyperlink{frame:otherX}{\beamergotobutton{back}}
\end{frame}

%%--------------------------------------------------------------%
%\begin{frame}
%\frametitle{Split by non-cognitive ability PGS}
%
%\begin{figure}[hbtp]
%\centering
%\includegraphics[height=0.8\textheight]{../../3_output/shock_effects/robustness_noncogPGS_6070_100_cvplot.png}
%\label{fig:noncogPGS}
%\end{figure}
%\hyperlink{frame:otherX}{\beamergotobutton{back}}
%\end{frame}
%
%--------------------------------------------------------------%
\begin{frame}
\frametitle{Split by risk-seeking PGS}

\begin{figure}[hbtp]
\centering
\includegraphics[height=0.8\textheight]{../../3_output/shock_effects/robustness_riskPGS_6070_100_cvplot.png}
\label{fig:riskPGS}
\end{figure}
\hyperlink{frame:otherX}{\beamergotobutton{back}}
\end{frame}

%%--------------------------------------------------------------%
%\begin{frame}
%\frametitle{Split by educational attainment PGS}
%
%\begin{figure}[hbtp]
%\centering
%\includegraphics[height=0.8\textheight]{../../3_output/shock_effects/robustness_ea3PGS_6070_100_cvplot.png}
%\label{fig:ea3PGS}
%\end{figure}
%\hyperlink{frame:otherX}{\beamergotobutton{back}}
%\end{frame}
%


















%--------------------------------------------------------------%
\begin{frame}[allowframebreaks]
\frametitle{References}
\begin{tiny}
\bibliographystyle{apalike}
%\bibliography{C:/Users/pbiroli/Documents/Papers/library}  %from laptop
% source picture: https://ululalbab31n.blogspot.com/2017/01/single-nucleotide-polymorphism.html
\bibliography{../health_genes}
\end{tiny}
\end{frame}

\end{document}
